\subsubsection{Escolha do tipo de estação de controle}\index{Escolha do tipo de estação de controle}

Dependendo das necessidades da operação, uma decisão importante que se deve tomar é se o local de apoio será fixo ou móvel. Estações fixas geralmente são postas no centro da cidade ou são construídas em pontos estratégicos para estabelecer a comunicação com o VANT. As bases móveis, pelo contrário, podem acompanhar o deslocamento do VANT durante a missão. Cada tipo de estação possui suas características:

\begin{table}[H]
\centering
\begin{tabular}{|l|p{4cm}|p{4cm}|}
\hline
\textbf{Característica} & \textbf{Estação Fixa} & \textbf{Base Móvel} \\ \hline
Transmissão & Grande alcance de transmissão. & Pequeno e médio alcance de transmissão. \\ \hline
Posicionamento & Centralizado. Pode ser colocada em pontos estratégicos. & Acompanha o VANT durante toda operação. \\ \hline
Mobilidade & Restrita pelo alcance do VANT. & Acesso rápido ao local que será monitorado. \\ \hline
Espaço & Possui maior capacidade (equipamentos, pessoas). & Restringido pelo modelo escolhido. \\ \hline
Recarga (VANT) & Demanda mais tempo. & Rápida. \\ \hline
Reposição (VANT) & Demanda mais tempo. & Rápida. \\ \hline
\end{tabular}
\caption{Características de estações}
\label{carac-estac}
\end{table}

A escolha do tipo de estação de controle foi baseada no estudo das características listadas acima e no objetivo que o VANT deve alcançar: apoio para operações policiais no monitoramento de manifestações. Percebe-se que os protestos geralmente se iniciam em locais estratégicos, mas podem ser imprevisíveis quanto a sua localização. Considerando que a polícia deve conter os focos com rapidez, a escolha de uma estação fixa não seria adequada por questões de posicionamento e mobilidade, visto que sua área de atuação seria rápida somente em locais estratégicos. Além disso, a estação fixa pecaria quanto ao tempo de recarga e reposição dos VANTs. Desta forma, foi escolhido que a estação de controle será uma base móvel, pois melhor atende as necessidades do policial.

\subsubsection{Adaptação}\index{Adaptação}
A adaptação do veículo consistiria na retirada dos bancos de passageiros da parte traseira da van, restando apenas os dois dianteiros, do motorista e 	um de passageiro.

Toda a parte de trás do veículo será utilizada como estação de controle e armazenamento dos VANTs, que contará com novos assentos, adaptados à função de posto de trabalho. Os painéis, telas, e dispositivos também serão dispostos de forma que facilitem  o trabalho dos operadores, de acordo com as regras da ergonomia e deixando livre o maior espaço possível para  a  rotina de trabalho.

Será feita uma abertura de 1,2 metros quadrado no teto do veículo, para disposição de um teto solar, que auxiliará a manipulação do VANT e permitirá sua decolagem vertical de dentro do carro.

Para a proteção dos equipamentos e mão de obra especializada dentro do veículo, são necessários itens básicos de segurança, como air bags e cintos de segurança. Portanto, a segurança do passageiro será readaptada ao novo interior veicular.
