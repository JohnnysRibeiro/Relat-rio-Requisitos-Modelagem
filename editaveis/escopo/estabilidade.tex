Na busca de uma melhor estabilidade de voo para garantir uma precisão e nitidez maior por parte da camêra precisamos integrar ao VANT componentes que tem a função de melhorar a capacidade de se manter parado no ar visto que o modelo em questão é um Hexacóptero.

Para garantir a estabilidade do VANT faz-se necessário o uso de alguns componentes que executem essa função, entre eles é possível citar:

\begin{itemize}
 \item Acelerômetro:  É um dispositivo que mede a vibração ou a aceleração do movimento de uma estrutura. A força causada por uma vibração ou alteração do movimento (aceleração) faz com que a massa "esprema" o material piezoelétrico, produzindo uma carga elétrica proporcional à força exercida sobre ele. Como a carga é proporcional à força e a massa é constante, a carga também é proporcional à aceleração.\cite{acelerometro2013}

 \item Giroscópio:  É um dispositivo usado para indicar as mudanças de direção de um objeto em movimento. No VANT será útil como instrumento de navegação, pois ajuda a manter  o mesmo em seu curso.\cite{giroscopio2013}

 \item Barômetro:  É um instrumento cuja função é medir a pressão atmosférica, obtendo a altitude do VANT, é usado para garantir que o mesmo não realize movimentos verticais indesejados.

 \item Sonar:  Medir a altitude do VANT, quando próximo ao solo, através de pulsos ultrassônicos.

 \item Magnetômetro: É um instrumento científico que mede os campos magnéticos. Além de determinar a força de um campo magnético, um magnetômetro também pode determinar orientação e direção de campos magnéticos.\cite{magnetometro2012}

\end{itemize}
