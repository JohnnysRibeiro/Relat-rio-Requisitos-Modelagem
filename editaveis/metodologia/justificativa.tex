
\textbf{Scrum}

O Scrum possui características que visam o melhor envolvimento da equipe e focam menos em documentação e mais em desenvolvimento do projeto. É dividido em períodos de tempo chamados \textit{sprints}, que duram de duas a quatro semanas. No começo da \textit{sprint}, faz-se uma reunião com a equipe e os \textit{stakeholders} para definir o planejamento da mesma. Nesse planejamento são traçadas as metas e tarefas que devem ser feitas durante a iteração. Diariamente, de preferência, faz-se uma reunião para o monitorar o acompanhamento do projeto. Além disso, os seguintes tópicos devem ser observados, quando a metodologia for utilizada: \cite{scrum2013}

\begin{itemize}
\item A discussão diária na qual cada membro da equipe responde às seguintes perguntas:
	\begin{itemize}
	\item O que fiz desde ontem?
	\item O que estou planejando fazer até amanhã?
	\item Existe algo me impedindo de atingir minha meta?
	\end{itemize}
\item Locais e horas de trabalho devem ser energizadas; (no sentido de que "trabalhar horas extras" não necessariamente significa "produzir mais".)
\item Transparência no planejamento e desenvolvimento;
\end{itemize}

\textbf{PMBoK}

O Project Management Body of Knowledge, também conhecido como PMBoK é um conjunto de práticas em gerência de projetos levantado pelo Project Management Institute (PMI) e constituem a base da metodologia de gerência de projetos do PMI.\cite{gomez2010}

Existe uma divisão entre tarefas e fases muito bem definidas no PMBoK, que, apesar de ser muito boa para projetos grandes e com um grande equipe, carece de uma relação mais forte com \textit{stakeholders} e a equipe, em geral.

O grupo decidiu escolher, a partir das características apresentadas, utilizar as partes do Scrum relativas às \textit{sprints} e às reuniões de acompanhamento do desenvolvimento do projeto. Apesar das discussões não terem sido diárias, elas foram periódicas e frequentes. A parte escolhida do PMBoK é relacionada com a produção e detalhamento dos artefatos necessários para o projeto, como a EAP e um detalhamento maior do cronograma.