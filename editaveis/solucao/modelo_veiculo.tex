\subsubsection{Escolha do tipo de estação de controle}\index{Escolha do tipo de estação de controle}

Dependendo das necessidades da operação, uma decisão importante que se deve tomar é se o local de apoio será fixo ou móvel. Estações fixas geralmente são postas no centro da cidade ou são construídas em pontos estratégicos para estabelecer a comunicação com o VANT. As bases móveis, pelo contrário, podem acompanhar o deslocamento do VANT durante a missão. Cada tipo de estação possui suas características:

\begin{table}[H]
\centering
\begin{tabular}{|l|p{4cm}|p{4cm}|}
\hline
\textbf{Característica} & \textbf{Estação Fixa} & \textbf{Base Móvel} \\ \hline
Transmissão & Grande alcance de transmissão. & Pequeno e médio alcance de transmissão. \\ \hline
Posicionamento & Centralizado. Pode ser colocada em pontos estratégicos. & Acompanha o VANT durante toda operação. \\ \hline
Mobilidade & Restrita pelo alcance do VANT. & Acesso rápido ao local que será monitorado. \\ \hline
Espaço & Possui maior capacidade (equipamentos, pessoas). & Restringido pelo modelo escolhido. \\ \hline
Recarga (VANT) & Demanda mais tempo. & Rápida. \\ \hline
Reposição (VANT) & Demanda mais tempo. & Rápida. \\ \hline
\end{tabular}
\caption{Características de estações}
\label{carac-estac}
\end{table}

A escolha do tipo de estação de controle foi baseada no estudo das características listadas acima e no objetivo que o VANT deve alcançar: apoio para operações policiais no monitoramento de manifestações. Percebe-se que os protestos geralmente se iniciam em locais estratégicos, mas podem ser imprevisíveis quanto a sua localização. Considerando que a polícia deve conter os focos com rapidez, a escolha de uma estação fixa não seria adequada por questões de posicionamento e mobilidade, visto que sua área de atuação seria rápida somente em locais estratégicos. Além disso, a estação fixa pecaria quanto ao tempo de recarga e reposição dos VANTs. Desta forma, foi escolhido que a estação de controle será uma base móvel, pois melhor atende as necessidades do policial.

\subsubsection{Análise de mercado das frotas policiais no Brasil}\index{Análise de mercado das frotas policiais no Brasil}

Vários modelos de carros são negociados com o governo para auxiliar a polícia brasileira em suas missões. No entanto, a escolha errada do veículo pode acarretar no fracasso da operação. Desta forma, já existe no mercado uma relação entre as categorias de carros usados pela policia para cada objetivo que se pode enfrentar:

\begin{itemize}
\item \textbf{Patrulhamento urbano}: são mais utilizados compactos populares, sedãs e hatches médios. As peruas são usadas quando há necessidade de maior espaço para os ocupantes e pessoas detidas.
\item \textbf{Patrulhamento de fronteiras e locais de difícil acesso (terrenos acidentados)}: nesta ocasião é mais comum se utilizar picapes e SUVs 4x4.
\item \textbf{Operações Táticas}: o uso de carros maiores, como blindados, SUVs 4x4 e vans, é ideal para o sucesso da missão.\cite{frotas2013}
\end{itemize}

Percebe-se também, que o avanço da tecnologia está cada vez maior e, desta forma, a polícia costuma inovar as suas frotas. A tabela a seguir, mostra alguns modelos que foram ou são usados pela polícia no Brasil:

%% fix gambiarra
\begin{table}[H]
\centering
\begin{tabular}{|p{3cm}|p{3cm}|p{3cm}|p{2cm}|p{1cm}|}
\hline
\textbf{Operação} & \textbf{Modelo} & \textbf{Força Policial} & \textbf{Local} & \textbf{Está em uso?} \\ \hline
Patrulhamento Urbano & VW Kombi & Polícia Feminina & SP & Não \\ \hline
 & VW Fusca & Polícia Rodoviária, Rádio Patrulha & Nacional & Não \\ \hline
 & VW Variant & Polícia Rodoviária & Nacional & Não \\ \hline
 & Chevrolet Monza Hatch & Polícia Rodoviária Estadual & SP & Não \\ \hline
 & Chevrolet Opala & Polícia Militar & SP & Não \\ \hline
 & GM Veraneio & PM & SP & Não \\ \hline
 & Ford Edge & Polícia Civil & SC & Sim \\ \hline
 & Renault Fluence & PM & PR & Sim \\ \hline
 & Chevrolet Vectra & Polícia Civil & DF & Sim \\ \hline
Patrulhamento em terrenos acidentados & Mitsubishi Pajero Dakar & Batalhão de Choque da PM & DF, MG, AM, GO & Sim \\ \hline
 & Nissan Frontier & BOPE, Força Nacional, PM & RJ & Sim \\ \hline
 & Chevrolet S10 & Polícia Rodoviária & Nacional & Sim \\ \hline
 & Mitsubishi L200 & Polícia Rodoviária & Nacional & Sim \\ \hline
 & Ford Ranger & PM & ES & Sim \\ \hline
Operações Táticas & Toyota SW4 & ROTA & SP & Sim \\ \hline
 & Chevrolet Blazer & ROTA, Polícia Civil & SP, RJ & Sim \\ \hline
 & Nissan Frontier & BOPE, Força Nacional, PM & RJ & Sim \\ \hline
 & Blindado & BOPE & RJ & Sim \\ \hline
 & Caveirão & BOPE & RJ & Sim \\ \hline
\end{tabular}
\caption{Veículos utilizados pela polícia militar no Brasil\cite{frotas2013}}
\end{table}

Esta tabela comprova que a polícia no Brasil está cada vez mais se adaptando com a necessidade da operação. Além disso, a polícia pode adaptar seus veículos, dando mais segurança aos soldados e agentes, ou seja, qualquer tipo de carro pode ser utilizado pela polícia, desde que seja útil na missão desejada.\cite{frotas2013}

\subsubsection{Escolha e adaptação do veículo terrestre para o VANT}\index{Escolha e adaptação do veículo terrestre para o VANT}

A estação de controle (CS ou \textit{Control Station}) é o centro do controle da operação homem-máquina. Pode ser instalada em terra (GCS - \textit{Ground Control Station}) ou em embarcações (SCS - \textit{Shipboard Control Station} ou ACS - \textit{Airborne Control Station}). Este trabalho será focado em SCS, pois a estação de controle será instalada em um veículo.

\subsubsection{Modelo do veículo}\index{Modelo do veículo}

A escolha do tipo de veículo foi baseada na necessidade de espaço interno , para armazenamento dos VANTs e aparelhos necessários para seu controle e observação, o que nos fez optar por um modelo furgão. Foram analisadas as fichas técnicas dos furgões presentes no mercado para chegar na melhor alternativa.

Dentre as opções analisadas no mercado brasileiro, o modelo escolhido foi o furgão Mercedez-Benz Sprinter 515 CDI, devido a seu maior comprimento e altura, tendo assim, maior dimensão interna e também em razão de sua alta capacidade de carregamento, o que permite carregar o peso necessário do aparelhamento e da tripulação com conforto e segurança.

\begin{figure}[H]
\centering\includegraphics[scale=0.5]{figuras/furgao_sprinter}
\caption{Mercedez-Benz Sprinter 515 CDI}
\end{figure}

\begin{figure}[H]
\centering\includegraphics[scale=0.5]{figuras/dados_tecnicos}
\caption{Dados técnicos do veículo escolhido}
\end{figure}

%%%%%%%%%%%%%%%%%%%%%%%%%%
\subsection{Adaptação}\index{Adaptação}

As particularidades do problema abordado neste projeto levou a ser necessário certas adaptações no veículo escolhido, e neste capítulo serão abordadas estas adaptações.
\subsubsection{Espaço interno}\index{Espaço interno}

Veículos utilitários de grandes dimensões possuem grande versatilidade, por isso são frequentemente utilizados como bases móveis, lojas, e até mesmo como \textit{food trucks} (espaço móvel que transporta e vende comida). A adaptação de furgões e vans para fins comerciais e corporativos é inovadora, e tem se apresentado com tendência no mercado brasileiro e internacional.\cite{foodtruckrace2014}

\begin{figure}[H]
\centering\includegraphics[scale=0.5]{figuras/food_truck}
\caption{\textit{Food truck}}
\end{figure}

No Brasil, existem empresas especializadas na adaptação de vans e furgões para diversas funcionalidades. Um exemplo é a Revescap\cite{revescap2014}, que trabalha para organizações governamentais(transformação de veículos em veículos de operações e  bases móveis) e para o setor privado.

\begin{figure}[H]
\centering\includegraphics[scale=0.5]{figuras/furgao_adaptado1}
\caption{Furgão adaptado}
\end{figure}

O veículo escolhido será utilizado como estação de controle e armazenamento dos VANTs, e contará com um posto de trabalho para o operador do VANT, compartimentos para o armazenamento da aparelhagem necessária, e espaço para o armazenamento dos VANTs. Os painéis, telas, e dispositivos serão dispostos de forma que facilitem o trabalho dos operadores, de acordo com as regras da ergonomia e deixando livre o maior espaço possível para a rotina de trabalho.

\begin{figure}[H]
\centering\includegraphics[scale=0.5]{figuras/sketch_furgao}
\caption{Interior da base móvel (cm)}
\end{figure}

\subsubsection{Segurança}\index{Segurança}

Afim de evitar o comprometimento do chassi, das peças de suporte, e consequentemente da estabilidade do veículo, todas as alterações feitas em seu interior respeitam o limite de cargas admissíveis sobre os eixos dianteiro e traseiro (1270 e 1243 toneladas, respectivamente).  Além disso, foram evitadas soldas e perfurações em colunas de sustentação da estrutura veicular, em pontos de incidência da carga, nos suportes dos eixos dianteiro e traseiro, em raios de torção e na zona dos airbags.\cite{diretivas2012}

% \subsubsection{Análise de custos}\index{Análise de custos}
