% ALIMENTAÇÃO VANT-CALCULOS DE ENERGIA ÚTIL E AUTONOMIA
\textbf{Bateria}

Atualmente as baterias mais utilizadas pelo aeromodelismo são baterias de Polímeros de Lítio (LiPo).  Isto se deve a sua alta capacidade de fornecer energia e serem muito mais leves que as demais existentes. Dentre várias fabricantes dessas baterias algumas das melhores e mais conhecidas são a Thunder Power e Hyperion. \cite{aeromodelismo2011}

Algumas especificidades devem ser levadas em conta na hora de optar pela escolha de uma bateria, pois ela deve suprir as demandas do projeto. As mais importantes são a carga elétrica (ou corrente) e a tensão. \cite{aeromodelismo2011} \cite{propulsao2011}

Considerando tais especificidades foi decidido que seriam utilizadas três  baterias LiPo. Duas para alimentar os motores e outra para alimentar os componentes eletrônicos de todo o projeto.

Os componentes eletrônicos do projeto necessitam de uma corrente de 2000mA e uma tensão de 5V para funcionarem perfeitamente. Com isso, dentre os vários tipos de baterias LiPo existentes, foi escolhido um modelo com uma corrente de 2000mAh e duas células conectadas em série, cada uma com tensão de 3,7 V, fornecendo então uma tensão total de 7,4V. A fabricante escolhida foi aThunder Power por ser uma empresa que monta baterias de altíssima qualidade, reconhecida no mercado mundial. Eles fornecem garantia de 2 anos em suas baterias e utilizam células Top Grade, que usam Lítio de altíssima pureza e um rigoroso controle de qualidade. A bateria de modelo: 2000mAh 2-CELL/2S 7.4VPRO LITE MS LIPO possui as especificações contidas na tabela abaixo. \cite{thunderpower2014}

\begin{table}[H]
\centering
\begin{tabular}{|l|l|}
\hline
\textbf{Max Charge} & 3C \\ \hline
\textbf{Max Charge Current} & 6A \\ \hline
\textbf{Max Cont. Discharge} & 16C \\ \hline
\textbf{Max Cont. Current} & 32A \\ \hline
\textbf{Max Burst} & 30C \\ \hline
\textbf{Max Burst Current} & 60A \\ \hline
\textbf{Weight} & 85g \\ \hline
\textbf{Dimensions (mm)} & 13x50x65 \\ \hline
\end{tabular}
\caption{Especificações da bateria para alimentação dos componentes eletrônicos \cite{thunderpower2014}}
\end{table}

Para os motores foi escolhida duas baterias LiPo com uma corrente de 6000mAh e duas células de 3,7V, fornecendo então uma tensão total de 7,4V,pois o motor demanda uma quantidade maiorde corrente. O modelo 6000mAh 2-CELL/2S3P 7.4V PRO LITE MS LIPO da fabricante Thunder Power possuias especificações contidas na tabela abaixo. \cite{thunderpower2014}

\begin{table}[H]
\centering
\begin{tabular}{|l|l|}
\hline
\textbf{Max Charge} & 3C \\ \hline
\textbf{Max Charge Current} & 6A \\ \hline
\textbf{Max Cont. Discharge} & 16C \\ \hline
\textbf{Max Cont. Current} & 32A \\ \hline
\textbf{Max Burst} & 30C \\ \hline
\textbf{Max Burst Current} & 60A \\ \hline
\textbf{Weight} & 85g \\ \hline
\textbf{Dimensions (mm)} & 13x50x65 \\ \hline
\end{tabular}
\caption{Especificações da bateria para alimentação dos motores \cite{thunderpower2014}}
\end{table}

As baterias escolhidas para alimentar os componentes eletrônicos (2000mAh) e motores (6000mAh) possuem valor estimado de U$ 55,61 e U$166,84 respectivamente. \cite{thunderpower2014}

%%figure Bateria para alimentação dos componentes eletrônicos http://www.amazon.com/Thunder-Power-RC-2000mAh-Battery/dp/B000PEP1J6

\begin{figure}[H]
\centering\includegraphics[scale=0.5]{figuras/bateria1}
\caption{Bateria para alimentação dos componentes eletrônicos}
\end{figure}

%%figure Bateria para alimentação dos motores http://www.draganflyhobbies.com/product/thunder-power-rc-pro-lite-ms-16c-6000mah-7-4v-2-cell-lipo-2s3pl-6000-lipo-battery/

\begin{figure}[H]
\centering\includegraphics[scale=0.5]{figuras/bateria2}
\caption{Bateria para alimentação dos motores}
\end{figure}

\textbf{Autonomia}

Os cálculos de autonomia para as baterias não são simples e necessitam de muitos parâmetros. Na literatura não há uma fórmula exata para calcular esses valores. Por isso são usadas diversas aproximações e parâmetros mais relevantes são determinados.

Três parâmetros principais que podem ser escolhidos e demonstrados: curva de descarga, a capacidade da bateria e a capacidade de descarga.

\begin{figure}[H]
\centering\includegraphics[scale=0.7]{figuras/curva_descarga}
\caption{Curva de Descarga \cite{riobotz2006}}
\end{figure}

A curva de descarga representa o decaimento da tensão ao longo do consumo da capacidade nominal. De acordo com a curva de descarga na figura anterior, verifica-se que esse fator não influencia diretamente nas baterias de lítio, pois possui uma curva de descarga quase horizontal e pode assegurar a voltagem desejada até próximo do consumo final.\cite{riobotz2006}

O segundo parâmetro é a capacidade da bateria. Ela quantifica o tempo para que ocorra uma descarga total da bateria. Sua medida usual é Ah (Ampère x hora). \cite{riobotz2006} Para os circuitos o uso de 2000mah, na teoria, seria uma duração de 1 hora, pois 2000ma/2000mah daria uma autonomia de 60 minutos de funcionamento. E para o motor, fazendo a mesma análise daria uma autonomia de 40 a 50 minutos. Porém o último fator influência diretamente o tempo.

A capacidade de descarga é quanto a bateria consegue fornecer sem que ocorra um superaquecimento dela, que poderia causar danos imensos a todo o projeto. Ela vem representada nas especificações pela letra C, como nas tabelas 1 e 2. \cite{riobotz2006}.Na tabela 1, há o valor de Max Burst, representado por 30C, com esse resultado, sem que ocorra o superaquecimento das células:

\begin{center}
30 C x 2000 mah = 60 A
\end{center}

Na prática, nos garante uma descarga no tempo mínimo de 1/30 horas, ou de 2 minutos, sem danificar o sistema de alimentação. Já para a bateria do motor seria 25 C x 6000 mah= 150 A, também sem que danifique a vida útil das baterias, porém com um tempo de 2,4 minutos. \cite{riobotz2006}

As recomendações do fabricante dos motores são a utilização de uma bateria de 5000mah. \cite{highpower2014} Porém, em reunião, foi decidido uma bateria de 6000mah. A justificativa vem do fato da escolha de possível ganho de aproximadamente 2,0 minutos, sem um aumento relevante de massa do VANT, além de uma conversa informal com os projetistas do VANT apresentado em sala. Foi explicado que os tempos de voo são geralmente determinados na prática, e que era atingido valores de 15 até 16 minutos com o VANT de oito hélices. \cite{nocoes2005}

Assim o tempo total de voo será definido principalmente pelo tempo de duração da bateria do motor, analisando além do fornecido pelo fabricante do motor, fatores como temperatura, capacidade de descarga, estima-se um voo de no máximo 18 minutos.
