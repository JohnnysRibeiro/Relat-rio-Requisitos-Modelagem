
\textbf{Modelos - Asas Rotativas}

Os modelos de asas rotativas podem ter de um até oito motores responsáveis pela sua sustentação e propulsão simultaneamente. As diferenças entre eles não são somente na quantidade de motores, e sim pela disposição deles no veículo, capacidade de carga suportada, o empuxo produzido,reação às falhas eàs condições climáticas.
Quando pesquisado os tipos de asas rotativas, foi desconsiderado o tipo mais clássico existente, o helicóptero, pelo fato do mesmo não ser estável quando está em voo, sua capacidade de carga irá depender do motor, logo como necessita-se de um veículo que suporte uma carga acima dos 2kg, ele necessitará de um motor mais potente e isso encareceria o projeto podendo o tornar inviável e por fim sua reação a condições climáticas adversas como vento forte pode atrapalhar a missão e comprometer o equipamento.
Com o helicóptero retirado das opções do grupo, restaram, então, o Quadricoptero (quatro motores) ou Hexacoptero (seis motores) ou o Octacoptero (oito motores). A disposição dos motores do quadricoptero podem ser dispostos na forma de X4 ou de I4, já o de seis motores pode ser na forma de Y6 ou de X6 ou IY6, enquanto o de oito motores na forma de um V8 ou X8 ou I8. A Imagem a seguir representa melhor as disposições dos motores:

Os modelos IY6, Y6, e X8 apesar de possuírem somente três ou quatro extremidades possuem dois motores em cada, isto é, um faz o empuxo e o outro traciona de forma que mantenha o vetor de impulso direcionado para baixo. A tabela a seguir apresenta os pontos a favor e os pontos contra de cada modelo, sendo que 4, 6 e 8 é relativo à quantidade de motores:

\begin{table}[H]
\centering
\begin{tabular}{c|p{6cm}|p{5cm}}
\toprule
\textbf{Modelo} & \textbf{Pontos positivos} & \textbf{Pontos Negativos} \\
\midrule
\textbf{I4} & Simples / Custo & Resistência ao clima \\ \hline
\textbf{X4} & Simples / Custo & Resistência condições climáticas \\ \hline
\textbf{I6} & Falhas não comprometem o voo / Capacidade de carga & Tamanho / Custo \\ \hline
\textbf{I6} & Falhas não comprometem o voo / Capacidade de carga & Tamanho / Custo \\ \hline
\textbf{X6} & Falhas não comprometem o voo / Capacidade de carga & Tamanho / Custo \\ \hline
\textbf{Y6} & Tamanho / Resistência ao clima & Eficiência / Complexidade Mecânica \\ \hline
\textbf{IY6}& Tamanho / Resistência ao clima & Eficiência / Complexidade Mecânica \\ \hline
\textbf{I8} & Alta resistência ao clima / Potência & Tamanho / Custo \\ \hline
\textbf{V8} & Alta resistência ao clima / Potência & Tamanho / Custo \\ \hline
\textbf{X8} & Alta Capacidade de carga & Ineficiência \\
\bottomrule
\end{tabular}
\caption{Comparação entre Modelos de Asas Rotativas}
\end{table}

\textbf{Quadricóptero x Hexacóptero}

Um quadricóptero possui uma mecânica mais simples, onde seu menor número de rotores ocasiona um consumo energético menor do que um VANT com mais rotores (um hexacóptero, por exemplo). Porém, este tipo e multirotor pode levar um peso menor de carga útil. Além disso, o quadricóptero não mantém o voo caso um de seus motores falhe.
Já o hexacóptero possui uma estabilidade de voo maior. Apesar de seu consumo energético ser maior, ele possibilita um armazenamento maior de energia e mais sustentação. Assim, ele pode carregar uma carga útil (como a câmera) com um peso maior do que o quadricóptero. Embora seu custo seja mais elevado em relação ao anterior, o hexacóptero tem a vantagem de continuar seu voo caso um de seus motores falhe. Sendo assim, escolheu-se o hexacóptero para ser utilizado no projeto.

\begin{table}
\begin{center}
\begin{tabular}{|c|c|c|} \hline
	Característica & Quadricóptero & Hexacóptero \\ \hline\hline
	Estabilidade do voo & & X \\ \hline
	Consumo energético & X & \\ \hline
	Armazenamento energético & & X \\ \hline
	Peso da carga útil & & X \\ \hline
	Custo & X & \\ \hline
	Segurança do voo & & X \\ \hline
\end{tabular}
\end{center}
\caption{Comparação características Quadri x Hexacóptero}
\end{table}

Então, para o uso do VANT em operações policiais, o tipo de veículo de asa rotatória escolhido foi o hexacóptero. Sua estabilidade de voo é favorável por fornecer uma melhor imagem e de forma mais rápida. Seu maior consumo energético é compensado por seu maior armazenamento de energia. Assim, juntamente com o seu número de rotores, a sustentação do VANT é maior, aumentando assim a capacidade de levar um peso maior de carga útil. Isso faz com que a escolha da câmera fique menos limitada devido ao peso. Por fim, sua capacidade de manter o voo quando um de seus motores falhar é essencial, pois evita que se perca o VANT (prejuízo operacional e econômico) e que esse atinja pessoas ou objetos (danos físicos.

Por ser um veículo que irá realizar uma fiscalização aérea, o mesmo não pode estar sujeito a falhas de motor, a necessidade de uma boa estabilidade não só em condições climáticas favoráveis como em condições adversas que ocorrem no DF (vento forte e chuva) e também a sua capacidade de carga, passou-se a desconsiderar todos os modelos de quatro rotores, portanto restando, na categoria de asas rotativas, os de seis e oito rotores.

Como havia restado o Hexacoptero e o Octacoptero que por sua vez possuem alto rendimento em relação a estabilidade e a capacidade de carga, decidimos descartar o e oito motores por não ser adaptável ao tamanho do “FURGÃO” e também por ser um veículo com excesso de rendimentos e consumo de energia por conta dos seus oito motores. E por conta disso acreditamos que ode seis motores irá se adaptar melhor à operação por conta do seu tamanho, sua estabilidade no momento do voo e na captura de imagens e também por conta de uma possível falha nos seus motores que não irá prejudicar a missão.

\textbf{Asa Fixa X Asa Rotativa}

No Brasil e no mundo já se faz o uso de vários VANTS para diversos objetivos, a força militar usa para monitoramento, polícia e bombeiros o usam como ferramenta para auxilio de operações de busca, de monitoramento, resgate, solução de crimes. Muitos agricultores também adquirem o VANT para monitoramento de suas terras e nos Estados Unidos foi bastante utilizado em guerras e eram compostos de armas. Com isso observa-se a existência de diversos tipos de VANTs, cada um com características especificas e que melhor se adéquam a uma missão. Os tipos principais são os Multirrotores e de Asa fixa. \cite{vantregras2013}

Os de asa fixa são mais aconselháveis para missões de monitoramento de grandes áreas, assim ele pode percorrer vários quilômetros por possuir uma grande autonomia, de uma hora ou mais em média. É uma estrutura mais complexa e menos versátil, já que seu pouso e decolagem dependem de uma pista adequada para ser lançado ao ar, o que limita suas ações em determinados locais. O seu vôo se assemelha à de um avião, em que as asas dão sustentação enquanto voa na vertical, por isso ele não pode ficar parado no ar, devendo assim manter o movimento vertical até o momento do pouso.

Os multirrotores são utilizados para monitoramento pontual de uma área, ou seja, de uma pequena área podendo pairar no ar assim conseguir imagens de certo local por mais tempo. É uma estrutura mais simples, em que sua propulsão depende de hélices que giram em sentidos opostos para que os torques se anulem permitindo um vôo mais estável. O vôo de um multirrotor se assemelha à de um helicóptero, em que as hélices realizam o movimento rotativo fazendo que o ar seja jogado para baixo, e em reação de mesma intensidade, porém de sentido oposto o helicóptero é lançado verticalmente. Possui menor autonomia, de 15 a 30 minutos em média, pois as baterias devem manter os motores elétricos funcionando a todo o momento, e graças às hélices o pouso e decolagem se dão na vertical, não necessitando de pistas de aterrissagem, assim seu pouso e decolagem podem ser feitas em lugares de difícil acesso. \cite{widmaier2005}

Para o objetivo desse projeto, o mais viável é o multirrotor, apesar de seu tempo de vôo ser pequeno, todas suas outras características são viáveis ao projeto. Dentro dos multirrotores têm-se os quadricoperos (4 motores), hexacopteros (6 motores) e octacopteros (8 motores).Diante das características, já citada, foi escolhido o X6 como o mais viável para o projeto.
